\chapter{DATABASE SERVER}
    \section{CONFIG}
        \subsection{db}
            \hspace*{0.6cm}Sử dụng mongoose để kết nối đến MongoDB thông qua MONGO\_URI từ tệp .env. Báo lỗi nếu kết nối thất bại và thoát process.
            \subsubsection{Import}
                \hspace*{0.6cm}Thư viện được nhập:
                \begin{lstlisting}
    import mongoose from 'mongoose';
                \end{lstlisting}
                \begin{itemize}
                    \item \texttt{mongoose}: Thư viện để kết nối và tương tác với cơ sở dữ liệu MongoDB.
                \end{itemize}

            \subsubsection{Hàm connectDB}
                \hspace*{0.6cm}Hàm thiết lập kết nối MongoDB:
                \begin{lstlisting}
    const connectDB = async () => {
        try {
            const conn = await mongoose.connect(process.env.MONGO_URI, {
            });

            console.log(`MongoDB connected: ${conn.connection.host}`);
        } catch (error) {
            console.error(`MongoDB connection error: ${error.message}`);
            process.exit(1);
        }
    };
                \end{lstlisting}
                \begin{itemize}
                    \item Lấy URI kết nối từ biến môi trường \texttt{MONGO\_URI}.
                    \item Sử dụng \texttt{mongoose.connect} để thiết lập kết nối với MongoDB.
                    \item Nếu thành công, ghi log địa chỉ host của kết nối.
                    \item Nếu thất bại, ghi log lỗi và thoát ứng dụng với mã lỗi 1.
                \end{itemize}

            \subsubsection{Export}
                \hspace*{0.6cm}Xuất hàm để sử dụng trong ứng dụng:
                \begin{lstlisting}
    export default connectDB;
                \end{lstlisting}
                \begin{itemize}
                    \item Xuất \texttt{connectDB} để gọi khi khởi động ứng dụng Node.js.
                \end{itemize}

            \subsubsection{Chức Năng Chính}
                \hspace*{0.6cm}Hàm \texttt{connectDB} cung cấp kết nối cơ sở dữ liệu:
                \begin{itemize}
                    \item \textbf{Kết Nối MongoDB}: Sử dụng \texttt{mongoose} để thiết lập kết nối với cơ sở dữ liệu MongoDB dựa trên \texttt{MONGO\_URI}.
                    \item \textbf{Xử Lý Lỗi}: Ghi log và thoát ứng dụng nếu kết nối thất bại, đảm bảo ứng dụng không chạy khi thiếu cơ sở dữ liệu.
                    \item \textbf{Ghi Log Kết Nối}: Thông báo khi kết nối thành công với địa chỉ host.
                    \item \textbf{Tích Hợp Ứng Dụng}: Hàm được xuất để dễ dàng gọi trong tệp khởi tạo server Express.
                \end{itemize}
    \section{MODEL}
        \hspace*{0.6cm}Chứa các schema và logic dữ liệu, thuộc tầng Data Layer (Layered Architecture) hoặc Model (MVC).
        \subsection{Device}
            \hspace*{0.6cm}Schema cho thiết bị, định nghĩa cấu trúc dữ liệu và các phương thức liên quan đến thiết bị.
            \subsubsection{Import}
                \hspace*{0.6cm}Thư viện được nhập:
                \begin{lstlisting}
    import mongoose from 'mongoose';
                \end{lstlisting}
                \begin{itemize}
                    \item \texttt{mongoose}: Thư viện để định nghĩa schema và tương tác với MongoDB.
                \end{itemize}

            \subsubsection{Định Nghĩa Schema}
                \hspace*{0.6cm}Tạo schema cho tài liệu thiết bị:
                \begin{lstlisting}
    const deviceSchema = new mongoose.Schema({
    name: { type: String, required: true },
    status: { type: String, enum: ['on', 'off'], default: 'off' },
    updatedAt: { type: Date, default: Date.now }
    });
                \end{lstlisting}
                \begin{itemize}
                    \item \texttt{deviceSchema}: Schema xác định cấu trúc tài liệu thiết bị với các trường:
                    \begin{itemize}
                        \item \texttt{name}: Chuỗi, bắt buộc (\texttt{required: true}).
                        \item \texttt{status}: Chuỗi, chỉ nhận giá trị \texttt{'on'} hoặc \texttt{'off'}, mặc định \texttt{'off'}.
                        \item \texttt{updatedAt}: Ngày giờ, mặc định là thời gian hiện tại (\texttt{Date.now}).
                    \end{itemize}
                \end{itemize}

            \subsubsection{Định Nghĩa Model}
                \hspace*{0.6cm}Tạo model từ schema:
                \begin{lstlisting}
    const Device = mongoose.model('Device', deviceSchema, 'GateWay');
                \end{lstlisting}
                \begin{itemize}
                    \item \texttt{Device}: Model được tạo từ \texttt{deviceSchema}, liên kết với bộ sưu tập \texttt{GateWay} trong MongoDB.
                    \item Tên model \texttt{Device} dùng để truy vấn và thao tác với tài liệu trong bộ sưu tập.
                \end{itemize}

            \subsubsection{Export}
                \hspace*{0.6cm}Xuất model để sử dụng trong ứng dụng:
                \begin{lstlisting}
    export default Device;
                \end{lstlisting}
                \begin{itemize}
                    \item Xuất \texttt{Device} để sử dụng trong các module khác, ví dụ: trong các hàm API để truy vấn hoặc cập nhật thiết bị.
                \end{itemize}

            \subsubsection{Chức Năng Chính}
                \hspace*{0.6cm}Schema và model \texttt{Device} cung cấp mô hình dữ liệu cho thiết bị:
                \begin{itemize}
                    \item \textbf{Định Nghĩa Cấu Trúc Dữ Liệu}: Xác định các trường \texttt{name}, \texttt{status}, và \texttt{updatedAt} với ràng buộc và giá trị mặc định.
                    \item \textbf{Tương Tác Với MongoDB}: Cung cấp model \texttt{Device} để thực hiện các thao tác như tạo, đọc, cập nhật, xóa tài liệu trong bộ sưu tập \texttt{GateWay}.
                    \item \textbf{Ràng Buộc Dữ Liệu}: Đảm bảo \texttt{name} bắt buộc, \texttt{status} chỉ nhận \texttt{'on'} hoặc \texttt{'off'}, và \texttt{updatedAt} tự động cập nhật.
                    \item \textbf{Tích Hợp Ứng Dụng}: Model dễ dàng sử dụng trong các API để quản lý thiết bị.
                \end{itemize}
        \subsection{User}
            \hspace*{0.6cm}Schema cho người dùng, định nghĩa cấu trúc dữ liệu và các phương thức liên quan đến người dùng.
            \subsubsection{Import}
                \hspace*{0.6cm}Các thư viện được nhập:
                \begin{lstlisting}
    import mongoose from 'mongoose';
    import bcrypt from 'bcryptjs';
                \end{lstlisting}
                \begin{itemize}
                    \item \texttt{mongoose}: Thư viện để định nghĩa schema và tương tác với MongoDB.
                    \item \texttt{bcryptjs}: Thư viện để mã hóa và so sánh mật khẩu.
                \end{itemize}

            \subsubsection{Định Nghĩa Schema}
                \hspace*{0.6cm}Tạo schema cho tài liệu người dùng:
    \begin{lstlisting}
    const userSchema = new mongoose.Schema({
        username: { type: String, required: true, unique: true },
        email:    { type: String, required: true, unique: true },
        password: { type: String, required: true },
        avatar:   { type: String, default: null },
    });
                \end{lstlisting}
                \begin{itemize}
                    \item \texttt{userSchema}: Schema xác định cấu trúc tài liệu người dùng với các trường:
                    \begin{itemize}
                        \item \texttt{username}: Chuỗi, bắt buộc, duy nhất.
                        \item \texttt{email}: Chuỗi, bắt buộc, duy nhất.
                        \item \texttt{password}: Chuỗi, bắt buộc.
                        \item \texttt{avatar}: Chuỗi, mặc định \texttt{null}.
                    \end{itemize}
                \end{itemize}

            \subsubsection{Mã Hóa Mật Khẩu}
                \hspace*{0.6cm}Middleware mã hóa mật khẩu trước khi lưu:
                \begin{lstlisting}
    userSchema.pre("save", async function (next) {
        if (!this.isModified("password")) return next();
        this.password = await bcrypt.hash(this.password, 10);
        next();
    });
                \end{lstlisting}
                \begin{itemize}
                    \item \texttt{pre("save")}: Middleware chạy trước khi lưu tài liệu.
                    \item Kiểm tra nếu \texttt{password} không thay đổi, bỏ qua bằng \texttt{next()}.
                    \item Mã hóa \texttt{password} bằng \texttt{bcrypt.hash} với độ mạnh 10, lưu vào \texttt{this.password}.
                \end{itemize}

            \subsubsection{Phương Thức So Sánh Mật Khẩu}
                \hspace*{0.6cm}Thêm phương thức để so sánh mật khẩu:
                \begin{lstlisting}
    userSchema.methods.matchPassword = async function (enteredPassword) {
        return await bcrypt.compare(enteredPassword, this.password);
    };
                \end{lstlisting}
                \begin{itemize}
                    \item \texttt{matchPassword}: Phương thức so sánh mật khẩu nhập vào với mật khẩu đã mã hóa bằng \texttt{bcrypt.compare}.
                    \item Trả về \texttt{true} nếu khớp, \texttt{false} nếu không.
                \end{itemize}

            \subsubsection{Định Nghĩa Model}
                \hspace*{0.6cm}Tạo model từ schema:
                \begin{lstlisting}
    const User = mongoose.model("User", userSchema);
                \end{lstlisting}
                \begin{itemize}
                    \item \texttt{User}: Model được tạo từ \texttt{userSchema}, liên kết với bộ sưu tập \texttt{users} (mặc định) trong MongoDB.
                    \item Tên model \texttt{User} dùng để truy vấn và thao tác với tài liệu người dùng.
                \end{itemize}

            \subsubsection{Chức Năng Chính}
                \hspace*{0.6cm}Schema và model \texttt{User} cung cấp mô hình dữ liệu cho người dùng:
                \begin{itemize}
                    \item \textbf{Định Nghĩa Cấu Trúc Dữ Liệu}: Xác định các trường \texttt{username}, \texttt{email}, \texttt{password}, và \texttt{avatar} với ràng buộc và giá trị mặc định.
                    \item \textbf{Mã Hóa Mật Khẩu}: Tự động mã hóa \texttt{password} trước khi lưu bằng \texttt{bcrypt}.
                    \item \textbf{So Sánh Mật Khẩu}: Cung cấp phương thức \texttt{matchPassword} để xác minh mật khẩu khi đăng nhập.
                    \item \textbf{Tương Tác Với MongoDB}: Model \texttt{User} hỗ trợ tạo, đọc, cập nhật, xóa tài liệu trong bộ sưu tập \texttt{users}.
                    \item \textbf{Tích Hợp Ứng Dụng}: Model dễ dàng sử dụng trong các API để quản lý người dùng.
                \end{itemize}
    \section{SEED}
        \hspace*{0.6cm}Tạo dữ liệu mẫu cho cơ sở dữ liệu, giúp kiểm tra và phát triển ứng dụng mà không cần nhập liệu thủ công.
        \subsection{userSeed}
            \hspace*{0.6cm}Tạo dữ liệu mẫu cho người dùng, bao gồm tên người dùng, email, mật khẩu và avatar.
            \subsubsection{Import}
                \hspace*{0.6cm}Các thư viện và mô-đun được nhập:
                \begin{lstlisting}
    import mongoose from 'mongoose';
    import dotenv from 'dotenv';
    import bcrypt from 'bcryptjs';
    import User from '../models/User.js';
    import path from 'path';
    import { fileURLToPath } from 'url';
                \end{lstlisting}
                \begin{itemize}
                    \item \texttt{mongoose}: Thư viện để kết nối và tương tác với MongoDB.
                    \item \texttt{dotenv}: Tải biến môi trường từ tệp \texttt{.env}.
                    \item \texttt{bcryptjs}: Thư viện để mã hóa mật khẩu (dùng trong model \texttt{User}).
                    \item \texttt{User}: Model người dùng từ \texttt{../models/User.js}.
                    \item \texttt{path}, \texttt{fileURLToPath}: Xử lý đường dẫn tệp.
                \end{itemize}

            \subsubsection{Biến Đường Dẫn}
                \hspace*{0.6cm}Định nghĩa đường dẫn tệp hiện tại:
                \begin{lstlisting}
    const __filename = fileURLToPath(import.meta.url);
    const __dirname = path.dirname(__filename);
                \end{lstlisting}
                \begin{itemize}
                    \item \texttt{\_\_filename}, \texttt{\_\_dirname}: Đường dẫn tệp hiện tại, hỗ trợ mô-đun ES.
                \end{itemize}

            \subsubsection{Cấu Hình Môi Trường}
                \hspace*{0.6cm}Tải biến môi trường từ tệp \texttt{.env}:
                \begin{lstlisting}
    dotenv.config({ path: path.join(__dirname, '../../.env') });
                \end{lstlisting}
                \begin{itemize}
                    \item \texttt{dotenv.config}: Tải các biến môi trường từ tệp \texttt{.env} trong thư mục gốc, cách thư mục hiện tại hai cấp.
                \end{itemize}

            \subsubsection{Kết Nối MongoDB}
                \hspace*{0.6cm}Thiết lập kết nối với MongoDB:
                \begin{lstlisting}
    await mongoose.connect(process.env.MONGO_URI);
                \end{lstlisting}
                \begin{itemize}
                    \item Sử dụng \texttt{mongoose.connect} để kết nối với MongoDB, lấy URI từ biến môi trường \texttt{MONGO\_URI}.
                \end{itemize}

            \subsubsection{Khởi Tạo Dữ Liệu}
                \hspace*{0.6cm}Xóa dữ liệu cũ và tạo người dùng mẫu:
                \begin{lstlisting}
    await User.deleteMany();

    await User.create({
        username: 'Admin',
        email: 'du.vohuudu@gmail.com',
        password: '123456',
        avatar: null,
    });

    console.log('Seeded user successfully!');
    process.exit();
                \end{lstlisting}
                \begin{itemize}
                    \item \texttt{User.deleteMany()}: Xóa tất cả tài liệu trong bộ sưu tập \texttt{users}.
                    \item \texttt{User.create}: Tạo người dùng mẫu với thông tin \texttt{username}, \texttt{email}, \texttt{password}, và \texttt{avatar}.
                    \item Ghi log thông báo thành công và thoát ứng dụng bằng \texttt{process.exit()}.
                \end{itemize}

            \subsubsection{Chức Năng Chính}
                \hspace*{0.6cm}Tệp khởi tạo dữ liệu cung cấp cơ chế thiết lập dữ liệu ban đầu:
                \begin{itemize}
                    \item \textbf{Kết Nối MongoDB}: Thiết lập kết nối với cơ sở dữ liệu MongoDB bằng \texttt{mongoose} và \texttt{MONGO\_URI}.
                    \item \textbf{Xóa Dữ Liệu Cũ}: Xóa toàn bộ tài liệu người dùng hiện có trong bộ sưu tập \texttt{users}.
                    \item \textbf{Tạo Người Dùng Mẫu}: Thêm một người dùng \texttt{Admin} với thông tin mẫu để khởi tạo dữ liệu.
                    \item \textbf{Quản Lý Môi Trường}: Tải biến môi trường từ tệp \texttt{.env} để cấu hình kết nối.
                    \item \textbf{Ghi Log và Thoát}: Thông báo khi hoàn tất và thoát ứng dụng.
                \end{itemize}
    \section{SERVER}
        \hspace*{0.6cm}Data Layer, xử lý dữ liệu và cung cấp API cho mainServer.Kết nối MongoDB, định nghĩa schema.
        \subsection{Import}
            \hspace*{0.6cm}Các thư viện và mô-đun được nhập:
            \begin{lstlisting}
    import express from "express";
    import dotenv from "dotenv";
    import connectDB from "./config/db.js";
    import User from "./models/User.js";
    import Device from "./models/Device.js";
    import path from "path";
    import { fileURLToPath } from "url";
            \end{lstlisting}
            \begin{itemize}
                \item \texttt{express}: Framework để tạo server và xử lý yêu cầu HTTP.
                \item \texttt{dotenv}: Tải biến môi trường từ tệp \texttt{.env}.
                \item \texttt{connectDB}: Hàm kết nối MongoDB từ \texttt{./config/db.js}.
                \item \texttt{User}, \texttt{Device}: Model MongoDB cho người dùng và thiết bị.
                \item \texttt{path}, \texttt{fileURLToPath}: Xử lý đường dẫn tệp.
            \end{itemize}

        \subsection{Biến Đường Dẫn}
            \hspace*{0.6cm}Định nghĩa đường dẫn tệp hiện tại:
            \begin{lstlisting}
    // Define __filename and __dirname
    const __filename = fileURLToPath(import.meta.url);
    const __dirname = path.dirname(__filename);
            \end{lstlisting}
            \begin{itemize}
                \item \texttt{\_\_filename}, \texttt{\_\_dirname}: Đường dẫn tệp hiện tại, hỗ trợ mô-đun ES.
            \end{itemize}

        \subsection{Cấu Hình Môi Trường}
            \hspace*{0.6cm}Tải biến môi trường từ tệp \texttt{.env}:
            \begin{lstlisting}
    // Read .env file from parent directory (Back-end)
    dotenv.config({ path: path.join(__dirname, "../.env") });
            \end{lstlisting}
            \begin{itemize}
                \item \texttt{dotenv.config}: Tải các biến môi trường từ tệp \texttt{.env} trong thư mục gốc, cách thư mục hiện tại một cấp.
            \end{itemize}

        \subsection{Kết Nối MongoDB}
            \hspace*{0.6cm}Gọi hàm kết nối cơ sở dữ liệu:
            \begin{lstlisting}
    // Call connectDB function
    connectDB();
            \end{lstlisting}
            \begin{itemize}
                \item \texttt{connectDB()}: Kết nối ứng dụng với MongoDB bằng cách gọi hàm từ \texttt{./config/db.js}.
            \end{itemize}

        \subsection{Khởi Tạo Server}
            \hspace*{0.6cm}Tạo và cấu hình server Express:
            \begin{lstlisting}
    const app = express();
    app.use(express.json());
            \end{lstlisting}
            \begin{itemize}
                \item \texttt{app}: Instance của \texttt{express} để xử lý yêu cầu HTTP.
                \item \texttt{express.json}: Middleware phân tích dữ liệu JSON từ yêu cầu.
            \end{itemize}

        \subsection{API Lấy Danh Sách Thiết Bị}
            \hspace*{0.6cm}Định nghĩa API GET để lấy tất cả thiết bị:
            \begin{lstlisting}
    app.get("/db/devices", async (req, res) => {
        try {
            const devices = await Device.find();
            res.json(devices);
        } catch (error) {
            res.status(500).json({ message: "Error fetching device list" });
        }
    });
            \end{lstlisting}
            \begin{itemize}
                \item Tuyến \texttt{/db/devices}: Truy vấn tất cả tài liệu thiết bị bằng \texttt{Device.find()}.
                \item Trả về danh sách thiết bị dưới dạng JSON hoặc lỗi 500 nếu thất bại.
            \end{itemize}

        \subsection{API Tìm Người Dùng Theo Email}
            \hspace*{0.6cm}Định nghĩa API GET để tìm người dùng bằng email:
            \begin{lstlisting}
    app.get("/db/users/email/:email", async (req, res) => {
        try {
            const user = await User.findOne({ email: req.params.email });
            if (!user) {
                return res.status(404).json({ message: "User not found" });
            }
            res.json(user);
        } catch (error) {
            res.status(500).json({ message: "Error fetching user" });
        }
    });
            \end{lstlisting}
            \begin{itemize}
                \item Tuyến \texttt{/db/users/email/:email}: Tìm người dùng bằng \texttt{User.findOne} với email.
                \item Trả về thông tin người dùng hoặc lỗi 404 nếu không tìm thấy, lỗi 500 nếu thất bại.
            \end{itemize}

        \subsection{API Tìm Người Dùng Theo ID}
            \hspace*{0.6cm}Định nghĩa API GET để tìm người dùng bằng ID:
            \begin{lstlisting}
    app.get("/db/users/:id", async (req, res) => {
        try {
            const user = await User.findById(req.params.id);
            if (!user) {
                return res.status(404).json({ message: "User not found" });
            }
            res.json(user);
        } catch (error) {
            res.status(500).json({ message: "Error fetching user" });
        }
    });
            \end{lstlisting}
            \begin{itemize}
                \item Tuyến \texttt{/db/users/:id}: Tìm người dùng bằng \texttt{User.findById} với ID.
                \item Trả về thông tin người dùng hoặc lỗi 404 nếu không tìm thấy, lỗi 500 nếu thất bại.
            \end{itemize}

        \subsection{API Tạo Người Dùng Mới}
            \hspace*{0.6cm}Định nghĩa API POST để tạo người dùng mới:
            \begin{lstlisting}
    app.post("/db/users", async (req, res) => {
        try {
            const user = await User.create(req.body);
            res.status(201).json(user);
        } catch (error) {
            res.status(500).json({ message: "Error creating user" });
        }
    });
            \end{lstlisting}
            \begin{itemize}
                \item Tuyến \texttt{/db/users}: Tạo người dùng mới bằng \texttt{User.create} với dữ liệu từ \texttt{req.body}.
                \item Trả về thông tin người dùng mới với mã 201 hoặc lỗi 500 nếu thất bại.
            \end{itemize}

        \subsection{API Cập Nhật Người Dùng}
            \hspace*{0.6cm}Định nghĩa API PUT để cập nhật người dùng:
            \begin{lstlisting}
    app.put("/db/users/:id", async (req, res) => {
        try {
            const user = await User.findByIdAndUpdate(req.params.id, req.body, { new: true });
            if (!user) {
                return res.status(404).json({ message: "User not found" });
            }
            res.json(user);
        } catch (error) {
            res.status(500).json({ message: "Error updating user" });
        }
    });
            \end{lstlisting}
            \begin{itemize}
                \item Tuyến \texttt{/db/users/:id}: Cập nhật người dùng bằng \texttt{User.findByIdAndUpdate} với ID và dữ liệu từ \texttt{req.body}.
                \item Trả về thông tin người dùng đã cập nhật hoặc lỗi 404 nếu không tìm thấy, lỗi 500 nếu thất bại.
            \end{itemize}

        \subsection{Khởi Động Server}
            \hspace*{0.6cm}Khởi động server trên cổng được chỉ định:
            \begin{lstlisting}
    const DB_PORT = process.env.DB_PORT || 5001;
    app.listen(DB_PORT, "0.0.0.0", () => {
        console.log(`Database Server running at http://0.0.0.0:${DB_PORT}`);
    });
            \end{lstlisting}
            \begin{itemize}
                \item \texttt{DB\_PORT}: Lấy từ biến môi trường hoặc mặc định 5001.
                \item \texttt{app.listen}: Khởi động server trên tất cả các giao diện mạng (\texttt{0.0.0.0}) tại cổng \texttt{DB\_PORT}.
            \end{itemize}

        \subsection{Chức Năng Chính}
            \hspace*{0.6cm}Tệp khởi tạo server cơ sở dữ liệu cung cấp các API để quản lý dữ liệu:
            \begin{itemize}
                \item \textbf{Kết Nối MongoDB}: Gọi \texttt{connectDB} để thiết lập kết nối với cơ sở dữ liệu MongoDB.
                \item \textbf{API Quản Lý Thiết Bị}: Cung cấp tuyến \texttt{/db/devices} để lấy tất cả thiết bị.
                \item \textbf{API Quản Lý Người Dùng}: Cung cấp các tuyến để tìm người dùng theo email hoặc ID, tạo và cập nhật người dùng.
                \item \textbf{Xử Lý Yêu Cầu JSON}: Sử dụng \texttt{express.json} để phân tích dữ liệu JSON từ yêu cầu.
                \item \textbf{Khởi Động Server}: Chạy server trên cổng \texttt{DB\_PORT} để xử lý các yêu cầu API.
            \end{itemize}


                    


