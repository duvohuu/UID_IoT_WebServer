\usepackage{vntex} % Tiếng Việt
\usepackage{graphicx} % Chèn hình ảnh
\usepackage{fancyhdr} % Gói hỗ trợ tạo header và footer fancy
\usepackage{changepage} % Thay đổi lề

% Chèn code
\usepackage{listings} % Thêm gói listings để chèn code
\usepackage{xcolor} % Màu cho code
% Định nghĩa màu sắc
\definecolor{codeblue}{RGB}{0, 0, 255}
\definecolor{codered}{RGB}{255, 0, 0}
\definecolor{codegray}{RGB}{128, 128, 128}
\definecolor{codeblack}{RGB}{0, 0, 0}
\definecolor{lightgray}{RGB}{245, 245, 245}
\definecolor{codecyan}{RGB}{0, 255, 255}

\lstdefinelanguage{javascript}{
    keywords={import, from, const, let, var, function, return, if, else, for, while, do, switch, case, break, continue, new, try, catch, throw, typeof, instanceof, void, with, async, await},
    keywordstyle=\color{codeblue}\bfseries,
    ndkeywords={class, export, default, extends, static, get, set},
    ndkeywordstyle=\color{codeblue}\bfseries,
    identifierstyle=\color{codeblack},
    sensitive=true,
    comment=[l]{//},
    morecomment=[s]{/*}{*/},
    commentstyle=\color{codegray}\itshape,
    stringstyle=\color{codered},
    morestring=[b]",
    morestring=[b]'
}

% Cấu hình listings với màu sắc
\lstset{
    basicstyle=\ttfamily\small,
    keywordstyle=\color{codeblue}\bfseries,
    stringstyle=\color{codered},
    commentstyle=\color{codegray}\itshape,
    numbers=left,
    numberstyle=\tiny\color{codegray},
    stepnumber=1,
    numbersep=5pt,
    showspaces=false,
    showstringspaces=false,
    frame=single,
    framerule=1pt,
    rulecolor=\color{codecyan},
    backgroundcolor=\color{lightgray},
    breaklines=true,
    breakatwhitespace=true,
    tabsize=2,
    captionpos=b,
    language=javascript
}

% Footnote and References
\usepackage[style=numeric,backend=biber]{biblatex} % Sử dụng gói biblatex
\usepackage{capt-of} %  Footnote trong caption
\usepackage[perpage]{footmisc} % Đánh số lại chú thích mỗi trang
\usepackage[toc,page]{appendix}

% Thiết lập bảng
\usepackage{array} % Gói hỗ trợ các bảng phức tạp
\usepackage{tabularx}
\usepackage{longtable} % Tạo bảng qua nhiều trang
\usepackage{cellspace}
\usepackage{diagbox} % Gói hỗ trợ tạo các ô chéo trong bảng
\usepackage{multirow}

% Thiết lập công thức toán học
\usepackage{amsmath} % Gói hỗ trợ các công thức toán học
\usepackage{amsfonts} % Gói hỗ trợ các ký hiệu toán học
\usepackage{amssymb} % Gói hỗ trợ các ký hiệu toán học
\usepackage{graphicx} % Gói hỗ trợ chèn hình ảnh
\usepackage{bm} % Chữ in đậm trong công thức toán 

% Thiết lập khác
\usepackage{tikz}
\usepackage{color}
\usepackage{subcaption}
\usepackage{framed}
\usepackage{float} % Để chèn hình ảnh vào đúng vị trí
\usepackage{fancyvrb} % Đưa dữ liệu dạng nguyên thủy vào


% Thiết lập kích thước
\usepackage{geometry}
\geometry{
    left=3cm,
    right=2cm,
    top=2.5cm,
    bottom=2.5cm,
}
\usepackage{hyperref} %Chèn link
\hypersetup{urlcolor=black,linkcolor=black,citecolor=black,colorlinks=true} % Màu cho các đường nét
\everymath{\color{black}}
\setlength{\headheight}{40pt}
\pagestyle{fancy}
